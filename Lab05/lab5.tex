% Options for packages loaded elsewhere
\PassOptionsToPackage{unicode}{hyperref}
\PassOptionsToPackage{hyphens}{url}
%
\documentclass[
]{article}
\usepackage{amsmath,amssymb}
\usepackage{iftex}
\ifPDFTeX
  \usepackage[T1]{fontenc}
  \usepackage[utf8]{inputenc}
  \usepackage{textcomp} % provide euro and other symbols
\else % if luatex or xetex
  \usepackage{unicode-math} % this also loads fontspec
  \defaultfontfeatures{Scale=MatchLowercase}
  \defaultfontfeatures[\rmfamily]{Ligatures=TeX,Scale=1}
\fi
\usepackage{lmodern}
\ifPDFTeX\else
  % xetex/luatex font selection
\fi
% Use upquote if available, for straight quotes in verbatim environments
\IfFileExists{upquote.sty}{\usepackage{upquote}}{}
\IfFileExists{microtype.sty}{% use microtype if available
  \usepackage[]{microtype}
  \UseMicrotypeSet[protrusion]{basicmath} % disable protrusion for tt fonts
}{}
\makeatletter
\@ifundefined{KOMAClassName}{% if non-KOMA class
  \IfFileExists{parskip.sty}{%
    \usepackage{parskip}
  }{% else
    \setlength{\parindent}{0pt}
    \setlength{\parskip}{6pt plus 2pt minus 1pt}}
}{% if KOMA class
  \KOMAoptions{parskip=half}}
\makeatother
\usepackage{xcolor}
\usepackage[margin=1in]{geometry}
\usepackage{graphicx}
\makeatletter
\def\maxwidth{\ifdim\Gin@nat@width>\linewidth\linewidth\else\Gin@nat@width\fi}
\def\maxheight{\ifdim\Gin@nat@height>\textheight\textheight\else\Gin@nat@height\fi}
\makeatother
% Scale images if necessary, so that they will not overflow the page
% margins by default, and it is still possible to overwrite the defaults
% using explicit options in \includegraphics[width, height, ...]{}
\setkeys{Gin}{width=\maxwidth,height=\maxheight,keepaspectratio}
% Set default figure placement to htbp
\makeatletter
\def\fps@figure{htbp}
\makeatother
\setlength{\emergencystretch}{3em} % prevent overfull lines
\providecommand{\tightlist}{%
  \setlength{\itemsep}{0pt}\setlength{\parskip}{0pt}}
\setcounter{secnumdepth}{-\maxdimen} % remove section numbering
\ifLuaTeX
  \usepackage{selnolig}  % disable illegal ligatures
\fi
\IfFileExists{bookmark.sty}{\usepackage{bookmark}}{\usepackage{hyperref}}
\IfFileExists{xurl.sty}{\usepackage{xurl}}{} % add URL line breaks if available
\urlstyle{same}
\hypersetup{
  pdftitle={lab5},
  pdfauthor={Hugo Morvan, Daniele Bozzoli},
  hidelinks,
  pdfcreator={LaTeX via pandoc}}

\title{lab5}
\author{Hugo Morvan, Daniele Bozzoli}
\date{2023-12-01}

\begin{document}
\maketitle

\hypertarget{question-1-hypothesis-testing}{%
\section{Question 1: Hypothesis
testing}\label{question-1-hypothesis-testing}}

\hypertarget{section}{%
\subsection{1.}\label{section}}

Create a scatterplot of Y versus X, are any patterns visible?

\hypertarget{section-1}{%
\subsection{2.}\label{section-1}}

Fit a curve to the data. First fit an ordinary linear model and then fit
and then one using loess(). Do these curves suggest that the lottery is
random? Explore how the resulting estimated curves are encoded and
whether it is possible to identify which parameters are responsible for
non--randomness.

\hypertarget{section-2}{%
\subsection{3.}\label{section-2}}

In order to check if the lottery is random, one can use various
statistics. One such possibility is based on the expected responses. The
fitted loess smoother provides an estimate \(\hat{Y}\) as a function of
X. It the lottery was random, we would expect \(\hat{Y}\) to be a flat
line, equaling the empirical mean of the observed responses, Y . The
statistic we will consider will be
\[ S = \sum_{i=1}^{n}|\hat{Y}_i-\bar{Y}| \]

If S is not close to zero, then this indicates some trend in the data,
and throws suspicion on the randomness of the lottery. Estimate S's
distribution through a non--parametric bootstrap, taking B = 2000
bootstrap samples. Decide if the lottery looks random, what is the
p--value of the observed value of S.

\hypertarget{section-3}{%
\subsection{4.}\label{section-3}}

We will now want to investigate the power of our considered test. First
based on the test statistic S, implement a function that tests the
hypothesis H0 : Lottery is random versus H1 : Lottery is non--random.
The function should return the value of S and its p--value, based on
2000 bootstrap samples.

\hypertarget{section-4}{%
\subsection{5.}\label{section-4}}

Now we will try to make a rough estimate of the power of the test
constructed in Step 4 by generating more and more biased samples:

\hypertarget{a}{%
\subsubsection{(a)}\label{a}}

Create a dataset of the same dimensions as the original data. Choose k,
out of the 366, dates and assign them the end numbers of the lottery
(i.e., they are not legible for the draw). The remaining 366 - k dates
should have random numbers assigned (from the set \{1, . . . , 366 -
k\}). The k dates should be chosen in two ways: \#\#\#\# i. k
consecutive dates, \#\#\#\# ii. as blocks (randomly scattered) of
\(\lfloor k/3 \rfloor\) consecutive dates (this is of course for
\(k ≥ 3\), and if k is not divisible by 3, then some blocks can be of
length \$ \lfloor k/3 \rfloor +1\$). \#\#\# (b) For each of the Plug the
two new not--completely--random datasets from item 5a into the bootstrap
test with B = 2000 and note whether it was rejected. \#\#\# (c) Repeat
Steps 5a--5b for k = 1, . . . , until you have observed a couple of
rejections.

How good is your test statistic at rejecting the null hypothesis of a
random lottery?

\end{document}
