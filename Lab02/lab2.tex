% Options for packages loaded elsewhere
\PassOptionsToPackage{unicode}{hyperref}
\PassOptionsToPackage{hyphens}{url}
%
\documentclass[
]{article}
\usepackage{amsmath,amssymb}
\usepackage{iftex}
\ifPDFTeX
  \usepackage[T1]{fontenc}
  \usepackage[utf8]{inputenc}
  \usepackage{textcomp} % provide euro and other symbols
\else % if luatex or xetex
  \usepackage{unicode-math} % this also loads fontspec
  \defaultfontfeatures{Scale=MatchLowercase}
  \defaultfontfeatures[\rmfamily]{Ligatures=TeX,Scale=1}
\fi
\usepackage{lmodern}
\ifPDFTeX\else
  % xetex/luatex font selection
\fi
% Use upquote if available, for straight quotes in verbatim environments
\IfFileExists{upquote.sty}{\usepackage{upquote}}{}
\IfFileExists{microtype.sty}{% use microtype if available
  \usepackage[]{microtype}
  \UseMicrotypeSet[protrusion]{basicmath} % disable protrusion for tt fonts
}{}
\makeatletter
\@ifundefined{KOMAClassName}{% if non-KOMA class
  \IfFileExists{parskip.sty}{%
    \usepackage{parskip}
  }{% else
    \setlength{\parindent}{0pt}
    \setlength{\parskip}{6pt plus 2pt minus 1pt}}
}{% if KOMA class
  \KOMAoptions{parskip=half}}
\makeatother
\usepackage{xcolor}
\usepackage[margin=1in]{geometry}
\usepackage{color}
\usepackage{fancyvrb}
\newcommand{\VerbBar}{|}
\newcommand{\VERB}{\Verb[commandchars=\\\{\}]}
\DefineVerbatimEnvironment{Highlighting}{Verbatim}{commandchars=\\\{\}}
% Add ',fontsize=\small' for more characters per line
\usepackage{framed}
\definecolor{shadecolor}{RGB}{248,248,248}
\newenvironment{Shaded}{\begin{snugshade}}{\end{snugshade}}
\newcommand{\AlertTok}[1]{\textcolor[rgb]{0.94,0.16,0.16}{#1}}
\newcommand{\AnnotationTok}[1]{\textcolor[rgb]{0.56,0.35,0.01}{\textbf{\textit{#1}}}}
\newcommand{\AttributeTok}[1]{\textcolor[rgb]{0.13,0.29,0.53}{#1}}
\newcommand{\BaseNTok}[1]{\textcolor[rgb]{0.00,0.00,0.81}{#1}}
\newcommand{\BuiltInTok}[1]{#1}
\newcommand{\CharTok}[1]{\textcolor[rgb]{0.31,0.60,0.02}{#1}}
\newcommand{\CommentTok}[1]{\textcolor[rgb]{0.56,0.35,0.01}{\textit{#1}}}
\newcommand{\CommentVarTok}[1]{\textcolor[rgb]{0.56,0.35,0.01}{\textbf{\textit{#1}}}}
\newcommand{\ConstantTok}[1]{\textcolor[rgb]{0.56,0.35,0.01}{#1}}
\newcommand{\ControlFlowTok}[1]{\textcolor[rgb]{0.13,0.29,0.53}{\textbf{#1}}}
\newcommand{\DataTypeTok}[1]{\textcolor[rgb]{0.13,0.29,0.53}{#1}}
\newcommand{\DecValTok}[1]{\textcolor[rgb]{0.00,0.00,0.81}{#1}}
\newcommand{\DocumentationTok}[1]{\textcolor[rgb]{0.56,0.35,0.01}{\textbf{\textit{#1}}}}
\newcommand{\ErrorTok}[1]{\textcolor[rgb]{0.64,0.00,0.00}{\textbf{#1}}}
\newcommand{\ExtensionTok}[1]{#1}
\newcommand{\FloatTok}[1]{\textcolor[rgb]{0.00,0.00,0.81}{#1}}
\newcommand{\FunctionTok}[1]{\textcolor[rgb]{0.13,0.29,0.53}{\textbf{#1}}}
\newcommand{\ImportTok}[1]{#1}
\newcommand{\InformationTok}[1]{\textcolor[rgb]{0.56,0.35,0.01}{\textbf{\textit{#1}}}}
\newcommand{\KeywordTok}[1]{\textcolor[rgb]{0.13,0.29,0.53}{\textbf{#1}}}
\newcommand{\NormalTok}[1]{#1}
\newcommand{\OperatorTok}[1]{\textcolor[rgb]{0.81,0.36,0.00}{\textbf{#1}}}
\newcommand{\OtherTok}[1]{\textcolor[rgb]{0.56,0.35,0.01}{#1}}
\newcommand{\PreprocessorTok}[1]{\textcolor[rgb]{0.56,0.35,0.01}{\textit{#1}}}
\newcommand{\RegionMarkerTok}[1]{#1}
\newcommand{\SpecialCharTok}[1]{\textcolor[rgb]{0.81,0.36,0.00}{\textbf{#1}}}
\newcommand{\SpecialStringTok}[1]{\textcolor[rgb]{0.31,0.60,0.02}{#1}}
\newcommand{\StringTok}[1]{\textcolor[rgb]{0.31,0.60,0.02}{#1}}
\newcommand{\VariableTok}[1]{\textcolor[rgb]{0.00,0.00,0.00}{#1}}
\newcommand{\VerbatimStringTok}[1]{\textcolor[rgb]{0.31,0.60,0.02}{#1}}
\newcommand{\WarningTok}[1]{\textcolor[rgb]{0.56,0.35,0.01}{\textbf{\textit{#1}}}}
\usepackage{graphicx}
\makeatletter
\def\maxwidth{\ifdim\Gin@nat@width>\linewidth\linewidth\else\Gin@nat@width\fi}
\def\maxheight{\ifdim\Gin@nat@height>\textheight\textheight\else\Gin@nat@height\fi}
\makeatother
% Scale images if necessary, so that they will not overflow the page
% margins by default, and it is still possible to overwrite the defaults
% using explicit options in \includegraphics[width, height, ...]{}
\setkeys{Gin}{width=\maxwidth,height=\maxheight,keepaspectratio}
% Set default figure placement to htbp
\makeatletter
\def\fps@figure{htbp}
\makeatother
\setlength{\emergencystretch}{3em} % prevent overfull lines
\providecommand{\tightlist}{%
  \setlength{\itemsep}{0pt}\setlength{\parskip}{0pt}}
\setcounter{secnumdepth}{-\maxdimen} % remove section numbering
\ifLuaTeX
  \usepackage{selnolig}  % disable illegal ligatures
\fi
\IfFileExists{bookmark.sty}{\usepackage{bookmark}}{\usepackage{hyperref}}
\IfFileExists{xurl.sty}{\usepackage{xurl}}{} % add URL line breaks if available
\urlstyle{same}
\hypersetup{
  pdftitle={732A90 Lab 2},
  pdfauthor={Daniele Bozzoli \& Hugo Morvan},
  hidelinks,
  pdfcreator={LaTeX via pandoc}}

\title{732A90 Lab 2}
\author{Daniele Bozzoli \& Hugo Morvan}
\date{2023-11-14}

\begin{document}
\maketitle

\hypertarget{question-1-optimisation-of-a-two-dimensional-function}{%
\subsection{Question 1: Optimisation of a two-dimensional
function}\label{question-1-optimisation-of-a-two-dimensional-function}}

\hypertarget{a}{%
\subsubsection{a)}\label{a}}

Countour plot:

\includegraphics{lab2_files/figure-latex/1-a-1.pdf}

\hypertarget{b}{%
\subsubsection{b)}\label{b}}

\begin{Shaded}
\begin{Highlighting}[]
\NormalTok{newton }\OtherTok{\textless{}{-}} \ControlFlowTok{function}\NormalTok{(x, epsilon)\{}
  \CommentTok{\#x is the starting vector}
\NormalTok{  dist }\OtherTok{\textless{}{-}} \DecValTok{999}
  
  \ControlFlowTok{while}\NormalTok{(dist }\SpecialCharTok{\textgreater{}}\NormalTok{ epsilon)\{}
\NormalTok{    gp }\OtherTok{\textless{}{-}} \FunctionTok{get\_gradient}\NormalTok{(x)}
\NormalTok{    gpp }\OtherTok{\textless{}{-}} \FunctionTok{get\_hessian}\NormalTok{(x)}
\NormalTok{    x\_next }\OtherTok{\textless{}{-}}\NormalTok{ x }\SpecialCharTok{{-}} \FunctionTok{solve}\NormalTok{(gpp)}\SpecialCharTok{\%*\%}\NormalTok{gp}
\NormalTok{    dist }\OtherTok{\textless{}{-}} \FunctionTok{sum}\NormalTok{((x}\SpecialCharTok{{-}}\NormalTok{x\_next)}\SpecialCharTok{*}\NormalTok{(x}\SpecialCharTok{{-}}\NormalTok{x\_next))}
\NormalTok{    x }\OtherTok{\textless{}{-}}\NormalTok{ x\_next}
\NormalTok{  \}}
  \FunctionTok{return}\NormalTok{(x)}
\NormalTok{\}}
\end{Highlighting}
\end{Shaded}

\hypertarget{c}{%
\subsubsection{c)}\label{c}}

Starting point : (x, y) = (2, 0), Newton method output : (1, -1).

Starting point : (x, y) = (-1, -2), Newton method output : (0, 1).

Starting point : (x, y) = (0, 1), Newton method output : (0, 1).

Starting point : (x, y) = (10, -10), Newton method output : (1, -1).

We obtain 2 points : (0,1) and (1,-1). The Gradient for the point (0,1)
is (0, 0) and its Hessian matrix is:

\begin{verbatim}
##      [,1] [,2]
## [1,]   -4   -2
## [2,]   -2    0
\end{verbatim}

The matrix is a negative semi-definite therefore this is a saddle point.

The Gradient for the point (1,-1) is (0, 0) and its Hessian matrix is:

\begin{verbatim}
##      [,1] [,2]
## [1,]   -4    2
## [2,]    2   -2
\end{verbatim}

The matrix is a negative definite therefore this is a local maximum.

\hypertarget{d}{%
\subsubsection{d)}\label{d}}

The disadvantage of using the steepest method instead of the Newton one
is that it converges slower, but the advantage is that the computation
of the Hessian matrix is not needed, which can be difficult sometimes to
calculate.

\hypertarget{question-2}{%
\subsection{Question 2:}\label{question-2}}

\hypertarget{a-1}{%
\subsubsection{a)}\label{a-1}}

\begin{Shaded}
\begin{Highlighting}[]
\NormalTok{X }\OtherTok{\textless{}{-}} \FunctionTok{c}\NormalTok{(}\DecValTok{0}\NormalTok{,}\DecValTok{0}\NormalTok{,}\DecValTok{0}\NormalTok{,}\FloatTok{0.1}\NormalTok{,}\FloatTok{0.1}\NormalTok{,}\FloatTok{0.3}\NormalTok{,}\FloatTok{0.3}\NormalTok{,}\FloatTok{0.9}\NormalTok{,}\FloatTok{0.9}\NormalTok{,}\FloatTok{0.9}\NormalTok{)}
\NormalTok{Y }\OtherTok{\textless{}{-}} \FunctionTok{c}\NormalTok{(}\DecValTok{0}\NormalTok{,}\DecValTok{0}\NormalTok{,}\DecValTok{1}\NormalTok{,}\DecValTok{0}\NormalTok{,}\DecValTok{1}\NormalTok{,}\DecValTok{1}\NormalTok{,}\DecValTok{1}\NormalTok{,}\DecValTok{0}\NormalTok{,}\DecValTok{1}\NormalTok{,}\DecValTok{1}\NormalTok{)}

\NormalTok{g }\OtherTok{\textless{}{-}} \ControlFlowTok{function}\NormalTok{(B)\{}
\NormalTok{  sum1 }\OtherTok{\textless{}{-}} \DecValTok{0}
  \ControlFlowTok{for}\NormalTok{ (q }\ControlFlowTok{in} \DecValTok{1}\SpecialCharTok{:}\DecValTok{10}\NormalTok{)}
\NormalTok{    sum1 }\OtherTok{\textless{}{-}}\NormalTok{ sum1 }\SpecialCharTok{+}\NormalTok{ (Y[q]}\SpecialCharTok{*}\FunctionTok{log}\NormalTok{((}\DecValTok{1}\SpecialCharTok{+}\FunctionTok{exp}\NormalTok{(}\SpecialCharTok{{-}}\NormalTok{B[}\DecValTok{1}\NormalTok{]}\SpecialCharTok{{-}}\NormalTok{B[}\DecValTok{2}\NormalTok{]}\SpecialCharTok{*}\NormalTok{X[q]))}\SpecialCharTok{\^{}{-}}\DecValTok{1}\NormalTok{)}\SpecialCharTok{+}\NormalTok{(}\DecValTok{1}\SpecialCharTok{{-}}\NormalTok{Y[q])}\SpecialCharTok{*}\FunctionTok{log}\NormalTok{(}\DecValTok{1}\SpecialCharTok{{-}}\NormalTok{(}\DecValTok{1}\SpecialCharTok{+}\FunctionTok{exp}\NormalTok{(}\SpecialCharTok{{-}}\NormalTok{B[}\DecValTok{1}\NormalTok{]}\SpecialCharTok{{-}}\NormalTok{B[}\DecValTok{2}\NormalTok{]}\SpecialCharTok{*}\NormalTok{X[q]))}\SpecialCharTok{\^{}{-}}\DecValTok{1}\NormalTok{))}
  \FunctionTok{return}\NormalTok{(sum1)}
\NormalTok{\}}

\NormalTok{grad }\OtherTok{\textless{}{-}} \ControlFlowTok{function}\NormalTok{(B)\{}
\NormalTok{  sum2 }\OtherTok{\textless{}{-}} \FunctionTok{c}\NormalTok{(}\DecValTok{0}\NormalTok{,}\DecValTok{0}\NormalTok{)}
  \ControlFlowTok{for}\NormalTok{ (i }\ControlFlowTok{in} \DecValTok{1}\SpecialCharTok{:}\DecValTok{10}\NormalTok{)}
\NormalTok{    sum2 }\OtherTok{\textless{}{-}}\NormalTok{ sum2 }\SpecialCharTok{+}\NormalTok{ ((Y[i] }\SpecialCharTok{{-}}\NormalTok{ (}\DecValTok{1}\SpecialCharTok{/}\NormalTok{(}\DecValTok{1}\SpecialCharTok{+}\FunctionTok{exp}\NormalTok{(}\SpecialCharTok{{-}}\NormalTok{B[}\DecValTok{1}\NormalTok{]}\SpecialCharTok{{-}}\NormalTok{B[}\DecValTok{2}\NormalTok{]}\SpecialCharTok{*}\NormalTok{X[i])))) }\SpecialCharTok{*} \FunctionTok{c}\NormalTok{(}\DecValTok{1}\NormalTok{,X[i]))}
  \FunctionTok{return}\NormalTok{(sum2)}
\NormalTok{\}}

\NormalTok{steepestasc }\OtherTok{\textless{}{-}} \ControlFlowTok{function}\NormalTok{(x0, }\AttributeTok{eps=}\FloatTok{1e{-}6}\NormalTok{, }\AttributeTok{alpha0=}\DecValTok{1}\NormalTok{)\{}
\NormalTok{  countfun }\OtherTok{\textless{}{-}} \DecValTok{0}
\NormalTok{  countgrad }\OtherTok{\textless{}{-}} \DecValTok{0}
\NormalTok{  xt   }\OtherTok{\textless{}{-}}\NormalTok{ x0}
\NormalTok{  conv }\OtherTok{\textless{}{-}} \DecValTok{999}
  \ControlFlowTok{while}\NormalTok{(conv }\SpecialCharTok{\textgreater{}}\NormalTok{ eps)\{ }
\NormalTok{    alpha }\OtherTok{\textless{}{-}}\NormalTok{ alpha0}
\NormalTok{    xt1   }\OtherTok{\textless{}{-}}\NormalTok{ xt}
\NormalTok{    xt    }\OtherTok{\textless{}{-}}\NormalTok{ xt1 }\SpecialCharTok{+}\NormalTok{ alpha}\SpecialCharTok{*}\FunctionTok{grad}\NormalTok{(xt1)}
\NormalTok{    countgrad }\OtherTok{\textless{}{-}}\NormalTok{ countgrad }\SpecialCharTok{+} \DecValTok{1}
    \ControlFlowTok{while}\NormalTok{ (}\FunctionTok{g}\NormalTok{(xt) }\SpecialCharTok{\textless{}} \FunctionTok{g}\NormalTok{(xt1))\{}
\NormalTok{      countfun }\OtherTok{\textless{}{-}}\NormalTok{ countfun }\SpecialCharTok{+} \DecValTok{2}
\NormalTok{      alpha }\OtherTok{\textless{}{-}}\NormalTok{ alpha}\SpecialCharTok{/}\DecValTok{2}
\NormalTok{      xt    }\OtherTok{\textless{}{-}}\NormalTok{ xt1 }\SpecialCharTok{+}\NormalTok{ alpha}\SpecialCharTok{*}\FunctionTok{grad}\NormalTok{(xt1)}
\NormalTok{      countgrad }\OtherTok{\textless{}{-}}\NormalTok{ countgrad }\SpecialCharTok{+} \DecValTok{1}
\NormalTok{    \}}
\NormalTok{    conv }\OtherTok{\textless{}{-}} \FunctionTok{sum}\NormalTok{((xt}\SpecialCharTok{{-}}\NormalTok{xt1)}\SpecialCharTok{*}\NormalTok{(xt}\SpecialCharTok{{-}}\NormalTok{xt1)) }\CommentTok{\# convergence criterion: distance between two successive iterates}
\NormalTok{  \}}
  \FunctionTok{cat}\NormalTok{(xt, }\StringTok{"}\SpecialCharTok{\textbackslash{}n}\StringTok{"}\NormalTok{)}
  \FunctionTok{cat}\NormalTok{(}\StringTok{"Function calls:"}\NormalTok{, countfun, }\StringTok{"}\SpecialCharTok{\textbackslash{}n}\StringTok{Gradiant calls:"}\NormalTok{, countgrad,}\StringTok{"}\SpecialCharTok{\textbackslash{}n}\StringTok{"}\NormalTok{)}
\NormalTok{\}}
\end{Highlighting}
\end{Shaded}

\hypertarget{b-1}{%
\subsubsection{b)}\label{b-1}}

Using Alpha = 1, the optimal point is found at :

\begin{verbatim}
## -0.008418083 1.260523 
## Function calls: 18 
## Gradiant calls: 28
\end{verbatim}

Using Alpha = 5, the optimal point is found at :

\begin{verbatim}
## -0.008803282 1.260276 
## Function calls: 114 
## Gradiant calls: 77
\end{verbatim}

We reach the same result, but with less function calls and gradient
calls for a backtracking starting value of Alpha = 1.

\hypertarget{c-1}{%
\subsubsection{c)}\label{c-1}}

optim function with Nelder-Mead method:

\begin{Shaded}
\begin{Highlighting}[]
\NormalTok{g2 }\OtherTok{\textless{}{-}} \ControlFlowTok{function}\NormalTok{(B)\{}
\NormalTok{  sum1 }\OtherTok{\textless{}{-}} \DecValTok{0}
  \ControlFlowTok{for}\NormalTok{ (q }\ControlFlowTok{in} \DecValTok{1}\SpecialCharTok{:}\DecValTok{10}\NormalTok{)}
\NormalTok{    sum1 }\OtherTok{\textless{}{-}}\NormalTok{ sum1 }\SpecialCharTok{+}\NormalTok{ (Y[q]}\SpecialCharTok{*}\FunctionTok{log}\NormalTok{((}\DecValTok{1}\SpecialCharTok{+}\FunctionTok{exp}\NormalTok{(}\SpecialCharTok{{-}}\NormalTok{B[}\DecValTok{1}\NormalTok{]}\SpecialCharTok{{-}}\NormalTok{B[}\DecValTok{2}\NormalTok{]}\SpecialCharTok{*}\NormalTok{X[q]))}\SpecialCharTok{\^{}{-}}\DecValTok{1}\NormalTok{)}\SpecialCharTok{+}\NormalTok{(}\DecValTok{1}\SpecialCharTok{{-}}\NormalTok{Y[q])}\SpecialCharTok{*}\FunctionTok{log}\NormalTok{(}\DecValTok{1}\SpecialCharTok{{-}}\NormalTok{(}\DecValTok{1}\SpecialCharTok{+}\FunctionTok{exp}\NormalTok{(}\SpecialCharTok{{-}}\NormalTok{B[}\DecValTok{1}\NormalTok{]}\SpecialCharTok{{-}}\NormalTok{B[}\DecValTok{2}\NormalTok{]}\SpecialCharTok{*}\NormalTok{X[q]))}\SpecialCharTok{\^{}{-}}\DecValTok{1}\NormalTok{))}
  \FunctionTok{return}\NormalTok{(}\SpecialCharTok{{-}}\NormalTok{sum1)}
\NormalTok{\}}

\FunctionTok{optim}\NormalTok{(}\FunctionTok{c}\NormalTok{(}\SpecialCharTok{{-}}\FloatTok{0.2}\NormalTok{, }\DecValTok{1}\NormalTok{), g2, }\AttributeTok{method=} \StringTok{"Nelder{-}Mead"}\NormalTok{)}
\end{Highlighting}
\end{Shaded}

\begin{verbatim}
## $par
## [1] -0.009423433  1.262738266
## 
## $value
## [1] 6.484279
## 
## $counts
## function gradient 
##       47       NA 
## 
## $convergence
## [1] 0
## 
## $message
## NULL
\end{verbatim}

optim function with BFGS method:

\begin{Shaded}
\begin{Highlighting}[]
\FunctionTok{optim}\NormalTok{(}\FunctionTok{c}\NormalTok{(}\SpecialCharTok{{-}}\FloatTok{0.2}\NormalTok{, }\DecValTok{1}\NormalTok{), g2, }\AttributeTok{method=} \StringTok{"BFGS"}\NormalTok{)}
\end{Highlighting}
\end{Shaded}

\begin{verbatim}
## $par
## [1] -0.009356112  1.262812883
## 
## $value
## [1] 6.484279
## 
## $counts
## function gradient 
##       12        8 
## 
## $convergence
## [1] 0
## 
## $message
## NULL
\end{verbatim}

The result is slightly different after 4 digits, but essentially the
same point. Our function is more precise because we can choose the
epsilon value for the stopping criterion. For reference, our function
used 18 Function calls and 28 Gradiant calls (for alpha = 1 and epsilon
= 10e-6). Nelder-Mead used 47 Function calls and 0 Gradiant calls (not
needed in the Nelder-Mead algorithm). BFGS used 12 Function calls and 8
Gradiant calls.

\hypertarget{d-1}{%
\subsubsection{d)}\label{d-1}}

\begin{Shaded}
\begin{Highlighting}[]
\FunctionTok{glm}\NormalTok{(Y}\SpecialCharTok{\textasciitilde{}}\NormalTok{X,}\AttributeTok{family =} \StringTok{"binomial"}\NormalTok{)}
\end{Highlighting}
\end{Shaded}

\begin{verbatim}
## 
## Call:  glm(formula = Y ~ X, family = "binomial")
## 
## Coefficients:
## (Intercept)            X  
##    -0.00936      1.26282  
## 
## Degrees of Freedom: 9 Total (i.e. Null);  8 Residual
## Null Deviance:       13.46 
## Residual Deviance: 12.97     AIC: 16.97
\end{verbatim}

We get the same result as the glm function. Great!

\hypertarget{appendix}{%
\subsection{Appendix:}\label{appendix}}

\hypertarget{a-2}{%
\subsubsection{1-a)}\label{a-2}}

\begin{Shaded}
\begin{Highlighting}[]
\NormalTok{f }\OtherTok{\textless{}{-}} \ControlFlowTok{function}\NormalTok{(p)\{}
\NormalTok{  x }\OtherTok{\textless{}{-}}\NormalTok{ p[}\DecValTok{1}\NormalTok{]}
\NormalTok{  y }\OtherTok{\textless{}{-}}\NormalTok{ p[}\DecValTok{2}\NormalTok{]}
  \FunctionTok{return}\NormalTok{(}\SpecialCharTok{{-}}\NormalTok{x}\SpecialCharTok{\^{}}\DecValTok{2} \SpecialCharTok{{-}}\NormalTok{x}\SpecialCharTok{\^{}}\DecValTok{2}\SpecialCharTok{*}\NormalTok{y}\SpecialCharTok{\^{}}\DecValTok{2} \SpecialCharTok{{-}}\DecValTok{2}\SpecialCharTok{*}\NormalTok{x}\SpecialCharTok{*}\NormalTok{y }\SpecialCharTok{+}\DecValTok{2}\SpecialCharTok{*}\NormalTok{x }\SpecialCharTok{+}\DecValTok{2}\NormalTok{)}
\NormalTok{\}}
\NormalTok{f\_x }\OtherTok{\textless{}{-}} \ControlFlowTok{function}\NormalTok{(p)\{}
\NormalTok{  x }\OtherTok{\textless{}{-}}\NormalTok{ p[}\DecValTok{1}\NormalTok{]}
\NormalTok{  y }\OtherTok{\textless{}{-}}\NormalTok{ p[}\DecValTok{2}\NormalTok{]}
  \FunctionTok{return}\NormalTok{(}\SpecialCharTok{{-}}\DecValTok{2}\SpecialCharTok{*}\NormalTok{x }\SpecialCharTok{{-}}\DecValTok{2}\SpecialCharTok{*}\NormalTok{x}\SpecialCharTok{*}\NormalTok{y}\SpecialCharTok{\^{}}\DecValTok{2} \SpecialCharTok{{-}}\DecValTok{2}\SpecialCharTok{*}\NormalTok{y }\SpecialCharTok{+}\DecValTok{2}\NormalTok{)}
\NormalTok{\}}
\NormalTok{f\_y }\OtherTok{\textless{}{-}} \ControlFlowTok{function}\NormalTok{(p)\{}
\NormalTok{  x }\OtherTok{\textless{}{-}}\NormalTok{ p[}\DecValTok{1}\NormalTok{]}
\NormalTok{  y }\OtherTok{\textless{}{-}}\NormalTok{ p[}\DecValTok{2}\NormalTok{]}
  \FunctionTok{return}\NormalTok{(}\SpecialCharTok{{-}}\DecValTok{2}\SpecialCharTok{*}\NormalTok{x}\SpecialCharTok{\^{}}\DecValTok{2}\SpecialCharTok{*}\NormalTok{y}\DecValTok{{-}2}\SpecialCharTok{*}\NormalTok{x)}
\NormalTok{\}}
\NormalTok{f\_xy }\OtherTok{\textless{}{-}} \ControlFlowTok{function}\NormalTok{(p)\{}
\NormalTok{  x }\OtherTok{\textless{}{-}}\NormalTok{ p[}\DecValTok{1}\NormalTok{]}
\NormalTok{  y }\OtherTok{\textless{}{-}}\NormalTok{ p[}\DecValTok{2}\NormalTok{]}
  \FunctionTok{return}\NormalTok{(}\SpecialCharTok{{-}}\DecValTok{4}\SpecialCharTok{*}\NormalTok{x}\SpecialCharTok{*}\NormalTok{y}\DecValTok{{-}2}\NormalTok{)}
\NormalTok{\}}
\NormalTok{f\_xx }\OtherTok{\textless{}{-}} \ControlFlowTok{function}\NormalTok{(p)\{}
\NormalTok{  x }\OtherTok{\textless{}{-}}\NormalTok{ p[}\DecValTok{1}\NormalTok{]}
\NormalTok{  y }\OtherTok{\textless{}{-}}\NormalTok{ p[}\DecValTok{2}\NormalTok{]}
  \FunctionTok{return}\NormalTok{(}\SpecialCharTok{{-}}\DecValTok{2} \SpecialCharTok{{-}}\DecValTok{2}\SpecialCharTok{*}\NormalTok{y}\SpecialCharTok{\^{}}\DecValTok{2}\NormalTok{)}
\NormalTok{\}}
\NormalTok{f\_yy }\OtherTok{\textless{}{-}} \ControlFlowTok{function}\NormalTok{(p)\{}
\NormalTok{  x }\OtherTok{\textless{}{-}}\NormalTok{ p[}\DecValTok{1}\NormalTok{]}
\NormalTok{  y }\OtherTok{\textless{}{-}}\NormalTok{ p[}\DecValTok{2}\NormalTok{]}
  \FunctionTok{return}\NormalTok{(}\SpecialCharTok{{-}}\DecValTok{2}\SpecialCharTok{*}\NormalTok{x}\SpecialCharTok{\^{}}\DecValTok{2}\NormalTok{)}
\NormalTok{\}}
\NormalTok{get\_gradient }\OtherTok{\textless{}{-}} \ControlFlowTok{function}\NormalTok{(p)\{}
  \FunctionTok{return}\NormalTok{(}\FunctionTok{c}\NormalTok{(}\FunctionTok{f\_x}\NormalTok{(p), }\FunctionTok{f\_y}\NormalTok{(p)))}
\NormalTok{\}}
\NormalTok{get\_hessian }\OtherTok{\textless{}{-}} \ControlFlowTok{function}\NormalTok{(p)\{}
\NormalTok{  hess }\OtherTok{\textless{}{-}} \FunctionTok{matrix}\NormalTok{(}\FunctionTok{c}\NormalTok{(}\FunctionTok{f\_xx}\NormalTok{(p), }\FunctionTok{f\_xy}\NormalTok{(p), }\FunctionTok{f\_xy}\NormalTok{(p), }\FunctionTok{f\_yy}\NormalTok{(p)), }\AttributeTok{ncol=}\DecValTok{2}\NormalTok{)}
  \FunctionTok{return}\NormalTok{(hess)}
\NormalTok{\}}

\NormalTok{x }\OtherTok{\textless{}{-}} \FunctionTok{seq}\NormalTok{(}\SpecialCharTok{{-}}\DecValTok{3}\NormalTok{,}\DecValTok{3}\NormalTok{,}\FloatTok{0.05}\NormalTok{)}
\NormalTok{y }\OtherTok{\textless{}{-}} \FunctionTok{seq}\NormalTok{(}\SpecialCharTok{{-}}\DecValTok{3}\NormalTok{,}\DecValTok{3}\NormalTok{,}\FloatTok{0.05}\NormalTok{)}

\NormalTok{lx }\OtherTok{\textless{}{-}} \FunctionTok{length}\NormalTok{(x)}
\NormalTok{ly }\OtherTok{\textless{}{-}} \FunctionTok{length}\NormalTok{(y)}

\NormalTok{points1 }\OtherTok{\textless{}{-}} \FunctionTok{matrix}\NormalTok{(}\DecValTok{0}\NormalTok{,}\AttributeTok{nrow=}\NormalTok{lx, }\AttributeTok{ncol=}\NormalTok{ly)}

\ControlFlowTok{for}\NormalTok{ (i }\ControlFlowTok{in} \DecValTok{1}\SpecialCharTok{:}\NormalTok{lx)\{}
  \ControlFlowTok{for}\NormalTok{ (q }\ControlFlowTok{in} \DecValTok{1}\SpecialCharTok{:}\NormalTok{ly)\{}
\NormalTok{    points1[i,q] }\OtherTok{\textless{}{-}} \FunctionTok{f}\NormalTok{(}\FunctionTok{c}\NormalTok{(x[i], y[q]))}
\NormalTok{  \}}
\NormalTok{\}}
\FunctionTok{contour}\NormalTok{(x,y,points1, }\AttributeTok{nlevels=}\DecValTok{50}\NormalTok{)}
\end{Highlighting}
\end{Shaded}

\includegraphics{lab2_files/figure-latex/A1-a-1.pdf}

\hypertarget{b-2}{%
\subsubsection{1-b)}\label{b-2}}

\begin{Shaded}
\begin{Highlighting}[]
\NormalTok{newton }\OtherTok{\textless{}{-}} \ControlFlowTok{function}\NormalTok{(x, epsilon)\{}
  \CommentTok{\#x is you starting vector}
\NormalTok{  dist }\OtherTok{\textless{}{-}} \DecValTok{999}
  
  \ControlFlowTok{while}\NormalTok{(dist }\SpecialCharTok{\textgreater{}}\NormalTok{ epsilon)\{}
\NormalTok{    gp }\OtherTok{\textless{}{-}} \FunctionTok{get\_gradient}\NormalTok{(x)}
\NormalTok{    gpp }\OtherTok{\textless{}{-}} \FunctionTok{get\_hessian}\NormalTok{(x)}
\NormalTok{    x\_next }\OtherTok{\textless{}{-}}\NormalTok{ x }\SpecialCharTok{{-}} \FunctionTok{solve}\NormalTok{(gpp)}\SpecialCharTok{\%*\%}\NormalTok{gp}
\NormalTok{    dist }\OtherTok{\textless{}{-}} \FunctionTok{sum}\NormalTok{((x}\SpecialCharTok{{-}}\NormalTok{x\_next)}\SpecialCharTok{*}\NormalTok{(x}\SpecialCharTok{{-}}\NormalTok{x\_next))}
\NormalTok{    x }\OtherTok{\textless{}{-}}\NormalTok{ x\_next}
\NormalTok{  \}}
  \FunctionTok{return}\NormalTok{(x)}
\NormalTok{\}}
\end{Highlighting}
\end{Shaded}

\hypertarget{c-2}{%
\subsubsection{1-c)}\label{c-2}}

\begin{Shaded}
\begin{Highlighting}[]
\NormalTok{p1 }\OtherTok{\textless{}{-}} \FunctionTok{c}\NormalTok{(}\DecValTok{2}\NormalTok{,}\DecValTok{0}\NormalTok{)}
\NormalTok{p2 }\OtherTok{\textless{}{-}} \FunctionTok{c}\NormalTok{(}\SpecialCharTok{{-}}\DecValTok{1}\NormalTok{,}\SpecialCharTok{{-}}\DecValTok{2}\NormalTok{)}
\NormalTok{p3 }\OtherTok{\textless{}{-}} \FunctionTok{c}\NormalTok{(}\DecValTok{0}\NormalTok{,}\DecValTok{1}\NormalTok{)}
\NormalTok{p4 }\OtherTok{\textless{}{-}} \FunctionTok{c}\NormalTok{(}\DecValTok{10}\NormalTok{,}\SpecialCharTok{{-}}\DecValTok{10}\NormalTok{)}
\NormalTok{epsilon }\OtherTok{\textless{}{-}} \FloatTok{1E{-}8}
\FunctionTok{newton}\NormalTok{(p1, epsilon)}
\end{Highlighting}
\end{Shaded}

\begin{verbatim}
##      [,1]
## [1,]    1
## [2,]   -1
\end{verbatim}

\begin{Shaded}
\begin{Highlighting}[]
\FunctionTok{newton}\NormalTok{(p2, epsilon)}
\end{Highlighting}
\end{Shaded}

\begin{verbatim}
##              [,1]
## [1,] 1.361391e-22
## [2,] 1.000000e+00
\end{verbatim}

\begin{Shaded}
\begin{Highlighting}[]
\FunctionTok{newton}\NormalTok{(p3, epsilon)}
\end{Highlighting}
\end{Shaded}

\begin{verbatim}
##      [,1]
## [1,]    0
## [2,]    1
\end{verbatim}

\begin{Shaded}
\begin{Highlighting}[]
\FunctionTok{newton}\NormalTok{(p4, epsilon)}
\end{Highlighting}
\end{Shaded}

\begin{verbatim}
##      [,1]
## [1,]    1
## [2,]   -1
\end{verbatim}

\begin{verbatim}
## [1] "Gradient and Hessian for the point (0,1)"
\end{verbatim}

\begin{verbatim}
## [1] 0 0
\end{verbatim}

\begin{verbatim}
##      [,1] [,2]
## [1,]   -4   -2
## [2,]   -2    0
\end{verbatim}

\begin{verbatim}
## [1] "this is a saddle point"
\end{verbatim}

\begin{verbatim}
## [1] "Gradient and Hessian for the point (1,-1)"
\end{verbatim}

\begin{verbatim}
## [1] 0 0
\end{verbatim}

\begin{verbatim}
##      [,1] [,2]
## [1,]   -4    2
## [2,]    2   -2
\end{verbatim}

\begin{verbatim}
## [1] "this is a maximum"
\end{verbatim}

\hypertarget{a-3}{%
\subsubsection{2-a)}\label{a-3}}

\begin{Shaded}
\begin{Highlighting}[]
\NormalTok{X }\OtherTok{\textless{}{-}} \FunctionTok{c}\NormalTok{(}\DecValTok{0}\NormalTok{,}\DecValTok{0}\NormalTok{,}\DecValTok{0}\NormalTok{,}\FloatTok{0.1}\NormalTok{,}\FloatTok{0.1}\NormalTok{,}\FloatTok{0.3}\NormalTok{,}\FloatTok{0.3}\NormalTok{,}\FloatTok{0.9}\NormalTok{,}\FloatTok{0.9}\NormalTok{,}\FloatTok{0.9}\NormalTok{)}
\NormalTok{Y }\OtherTok{\textless{}{-}} \FunctionTok{c}\NormalTok{(}\DecValTok{0}\NormalTok{,}\DecValTok{0}\NormalTok{,}\DecValTok{1}\NormalTok{,}\DecValTok{0}\NormalTok{,}\DecValTok{1}\NormalTok{,}\DecValTok{1}\NormalTok{,}\DecValTok{1}\NormalTok{,}\DecValTok{0}\NormalTok{,}\DecValTok{1}\NormalTok{,}\DecValTok{1}\NormalTok{)}

\NormalTok{g }\OtherTok{\textless{}{-}} \ControlFlowTok{function}\NormalTok{(B)\{}
\NormalTok{  sum1 }\OtherTok{\textless{}{-}} \DecValTok{0}
  \ControlFlowTok{for}\NormalTok{ (q }\ControlFlowTok{in} \DecValTok{1}\SpecialCharTok{:}\DecValTok{10}\NormalTok{)}
\NormalTok{    sum1 }\OtherTok{\textless{}{-}}\NormalTok{ sum1 }\SpecialCharTok{+}\NormalTok{ (Y[q]}\SpecialCharTok{*}\FunctionTok{log}\NormalTok{((}\DecValTok{1}\SpecialCharTok{+}\FunctionTok{exp}\NormalTok{(}\SpecialCharTok{{-}}\NormalTok{B[}\DecValTok{1}\NormalTok{]}\SpecialCharTok{{-}}\NormalTok{B[}\DecValTok{2}\NormalTok{]}\SpecialCharTok{*}\NormalTok{X[q]))}\SpecialCharTok{\^{}{-}}\DecValTok{1}\NormalTok{)}\SpecialCharTok{+}\NormalTok{(}\DecValTok{1}\SpecialCharTok{{-}}\NormalTok{Y[q])}\SpecialCharTok{*}\FunctionTok{log}\NormalTok{(}\DecValTok{1}\SpecialCharTok{{-}}\NormalTok{(}\DecValTok{1}\SpecialCharTok{+}\FunctionTok{exp}\NormalTok{(}\SpecialCharTok{{-}}\NormalTok{B[}\DecValTok{1}\NormalTok{]}\SpecialCharTok{{-}}\NormalTok{B[}\DecValTok{2}\NormalTok{]}\SpecialCharTok{*}\NormalTok{X[q]))}\SpecialCharTok{\^{}{-}}\DecValTok{1}\NormalTok{))}
  \FunctionTok{return}\NormalTok{(sum1)}
\NormalTok{\}}

\NormalTok{grad }\OtherTok{\textless{}{-}} \ControlFlowTok{function}\NormalTok{(B)\{}
\NormalTok{  sum2 }\OtherTok{\textless{}{-}} \FunctionTok{c}\NormalTok{(}\DecValTok{0}\NormalTok{,}\DecValTok{0}\NormalTok{)}
  \ControlFlowTok{for}\NormalTok{ (i }\ControlFlowTok{in} \DecValTok{1}\SpecialCharTok{:}\DecValTok{10}\NormalTok{)}
\NormalTok{    sum2 }\OtherTok{\textless{}{-}}\NormalTok{ sum2 }\SpecialCharTok{+}\NormalTok{ ((Y[i] }\SpecialCharTok{{-}}\NormalTok{ (}\DecValTok{1}\SpecialCharTok{/}\NormalTok{(}\DecValTok{1}\SpecialCharTok{+}\FunctionTok{exp}\NormalTok{(}\SpecialCharTok{{-}}\NormalTok{B[}\DecValTok{1}\NormalTok{]}\SpecialCharTok{{-}}\NormalTok{B[}\DecValTok{2}\NormalTok{]}\SpecialCharTok{*}\NormalTok{X[i])))) }\SpecialCharTok{*} \FunctionTok{c}\NormalTok{(}\DecValTok{1}\NormalTok{,X[i]))}
  \FunctionTok{return}\NormalTok{(sum2)}
\NormalTok{\}}

\NormalTok{steepestasc }\OtherTok{\textless{}{-}} \ControlFlowTok{function}\NormalTok{(x0, }\AttributeTok{eps=}\FloatTok{1e{-}8}\NormalTok{, }\AttributeTok{alpha0=}\DecValTok{1}\NormalTok{)}
\NormalTok{\{}
\NormalTok{  countfun }\OtherTok{\textless{}{-}} \DecValTok{0}
\NormalTok{  countgrad }\OtherTok{\textless{}{-}} \DecValTok{0}
\NormalTok{  xt   }\OtherTok{\textless{}{-}}\NormalTok{ x0}
\NormalTok{  conv }\OtherTok{\textless{}{-}} \DecValTok{999}
  \ControlFlowTok{while}\NormalTok{(conv }\SpecialCharTok{\textgreater{}}\NormalTok{ eps)}
\NormalTok{  \{}
\NormalTok{    alpha }\OtherTok{\textless{}{-}}\NormalTok{ alpha0}
\NormalTok{    xt1   }\OtherTok{\textless{}{-}}\NormalTok{ xt}
\NormalTok{    xt    }\OtherTok{\textless{}{-}}\NormalTok{ xt1 }\SpecialCharTok{+}\NormalTok{ alpha}\SpecialCharTok{*}\FunctionTok{grad}\NormalTok{(xt1)}
\NormalTok{    countgrad }\OtherTok{\textless{}{-}}\NormalTok{ countgrad }\SpecialCharTok{+} \DecValTok{1}
    \ControlFlowTok{while}\NormalTok{ (}\FunctionTok{g}\NormalTok{(xt) }\SpecialCharTok{\textless{}} \FunctionTok{g}\NormalTok{(xt1))}
\NormalTok{    \{}
\NormalTok{      countfun }\OtherTok{\textless{}{-}}\NormalTok{ countfun }\SpecialCharTok{+} \DecValTok{2}
\NormalTok{      alpha }\OtherTok{\textless{}{-}}\NormalTok{ alpha}\SpecialCharTok{/}\DecValTok{2}
\NormalTok{      xt    }\OtherTok{\textless{}{-}}\NormalTok{ xt1 }\SpecialCharTok{+}\NormalTok{ alpha}\SpecialCharTok{*}\FunctionTok{grad}\NormalTok{(xt1)}
\NormalTok{      countgrad }\OtherTok{\textless{}{-}}\NormalTok{ countgrad }\SpecialCharTok{+} \DecValTok{1}
\NormalTok{    \}}
\NormalTok{    conv }\OtherTok{\textless{}{-}} \FunctionTok{sum}\NormalTok{((xt}\SpecialCharTok{{-}}\NormalTok{xt1)}\SpecialCharTok{*}\NormalTok{(xt}\SpecialCharTok{{-}}\NormalTok{xt1))}
\NormalTok{  \}}
  \FunctionTok{cat}\NormalTok{(xt, }\StringTok{"}\SpecialCharTok{\textbackslash{}n}\StringTok{"}\NormalTok{)}
  \FunctionTok{cat}\NormalTok{(}\StringTok{"Function calls:"}\NormalTok{, countfun, }\StringTok{"}\SpecialCharTok{\textbackslash{}n}\StringTok{Gradiant calls:"}\NormalTok{, countgrad,}\StringTok{"}\SpecialCharTok{\textbackslash{}n}\StringTok{"}\NormalTok{)}
\NormalTok{\}}
\end{Highlighting}
\end{Shaded}

\hypertarget{b-3}{%
\subsubsection{2-b)}\label{b-3}}

Using Alpha = 1 :

\begin{Shaded}
\begin{Highlighting}[]
\FunctionTok{steepestasc}\NormalTok{(}\FunctionTok{c}\NormalTok{(}\SpecialCharTok{{-}}\FloatTok{0.2}\NormalTok{, }\DecValTok{1}\NormalTok{))}
\end{Highlighting}
\end{Shaded}

\begin{verbatim}
## -0.009316239 1.262656 
## Function calls: 26 
## Gradiant calls: 43
\end{verbatim}

Using Alpha = 5 :

\begin{Shaded}
\begin{Highlighting}[]
\FunctionTok{steepestasc}\NormalTok{(}\FunctionTok{c}\NormalTok{(}\SpecialCharTok{{-}}\FloatTok{0.2}\NormalTok{, }\DecValTok{1}\NormalTok{), }\AttributeTok{alpha0=}\DecValTok{5}\NormalTok{)}
\end{Highlighting}
\end{Shaded}

\begin{verbatim}
## -0.009327714 1.262664 
## Function calls: 180 
## Gradiant calls: 122
\end{verbatim}

We reach the same result, but with less function calls and gradient
calls.

\hypertarget{c-3}{%
\subsubsection{2-c)}\label{c-3}}

\begin{Shaded}
\begin{Highlighting}[]
\NormalTok{g2 }\OtherTok{\textless{}{-}} \ControlFlowTok{function}\NormalTok{(B)\{}
\NormalTok{  sum1 }\OtherTok{\textless{}{-}} \DecValTok{0}
  \ControlFlowTok{for}\NormalTok{ (q }\ControlFlowTok{in} \DecValTok{1}\SpecialCharTok{:}\DecValTok{10}\NormalTok{)}
\NormalTok{    sum1 }\OtherTok{\textless{}{-}}\NormalTok{ sum1 }\SpecialCharTok{+}\NormalTok{ (Y[q]}\SpecialCharTok{*}\FunctionTok{log}\NormalTok{((}\DecValTok{1}\SpecialCharTok{+}\FunctionTok{exp}\NormalTok{(}\SpecialCharTok{{-}}\NormalTok{B[}\DecValTok{1}\NormalTok{]}\SpecialCharTok{{-}}\NormalTok{B[}\DecValTok{2}\NormalTok{]}\SpecialCharTok{*}\NormalTok{X[q]))}\SpecialCharTok{\^{}{-}}\DecValTok{1}\NormalTok{)}\SpecialCharTok{+}\NormalTok{(}\DecValTok{1}\SpecialCharTok{{-}}\NormalTok{Y[q])}\SpecialCharTok{*}\FunctionTok{log}\NormalTok{(}\DecValTok{1}\SpecialCharTok{{-}}\NormalTok{(}\DecValTok{1}\SpecialCharTok{+}\FunctionTok{exp}\NormalTok{(}\SpecialCharTok{{-}}\NormalTok{B[}\DecValTok{1}\NormalTok{]}\SpecialCharTok{{-}}\NormalTok{B[}\DecValTok{2}\NormalTok{]}\SpecialCharTok{*}\NormalTok{X[q]))}\SpecialCharTok{\^{}{-}}\DecValTok{1}\NormalTok{))}
  \FunctionTok{return}\NormalTok{(}\SpecialCharTok{{-}}\NormalTok{sum1)}
\NormalTok{\}}

\FunctionTok{optim}\NormalTok{(}\FunctionTok{c}\NormalTok{(}\SpecialCharTok{{-}}\FloatTok{0.2}\NormalTok{, }\DecValTok{1}\NormalTok{), g2, }\AttributeTok{method=} \StringTok{"Nelder{-}Mead"}\NormalTok{)}
\end{Highlighting}
\end{Shaded}

\begin{verbatim}
## $par
## [1] -0.009423433  1.262738266
## 
## $value
## [1] 6.484279
## 
## $counts
## function gradient 
##       47       NA 
## 
## $convergence
## [1] 0
## 
## $message
## NULL
\end{verbatim}

\begin{Shaded}
\begin{Highlighting}[]
\FunctionTok{optim}\NormalTok{(}\FunctionTok{c}\NormalTok{(}\SpecialCharTok{{-}}\FloatTok{0.2}\NormalTok{, }\DecValTok{1}\NormalTok{), g2, }\AttributeTok{method=} \StringTok{"BFGS"}\NormalTok{)}
\end{Highlighting}
\end{Shaded}

\begin{verbatim}
## $par
## [1] -0.009356112  1.262812883
## 
## $value
## [1] 6.484279
## 
## $counts
## function gradient 
##       12        8 
## 
## $convergence
## [1] 0
## 
## $message
## NULL
\end{verbatim}

\hypertarget{d-2}{%
\subsubsection{2-d)}\label{d-2}}

\begin{Shaded}
\begin{Highlighting}[]
\FunctionTok{glm}\NormalTok{(Y}\SpecialCharTok{\textasciitilde{}}\NormalTok{X,}\AttributeTok{family =} \StringTok{"binomial"}\NormalTok{)}
\end{Highlighting}
\end{Shaded}

\begin{verbatim}
## 
## Call:  glm(formula = Y ~ X, family = "binomial")
## 
## Coefficients:
## (Intercept)            X  
##    -0.00936      1.26282  
## 
## Degrees of Freedom: 9 Total (i.e. Null);  8 Residual
## Null Deviance:       13.46 
## Residual Deviance: 12.97     AIC: 16.97
\end{verbatim}

\end{document}
